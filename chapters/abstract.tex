%!TEX program=luatex

\newpage

Avec l'émergence des TIC et la démocratisation de la production de contenu sur internet, un tsunami informationnel désormais très difficile pour être gérer manuellement ni même par l'outil informatique classique. Les journaux et les sites d'information contribuent pleinement dans ce contenu, ce qui rend la revue de presse quotidienne très difficile pour les lecteurs. Notre projet propose un outil sophistiqué qui prend en considération les centres d'interets des utilisateurs pour leurs suggérer les articles les plus pertinents de la presse algerienne ou internationnale dans les deux langues Arabe et Anglais en utilisant des techniques d'intelligence artificielle, de traitement automatique du langage et d'apprentissage automatique. 

Mots clés : Traitement automatique du langage naturel, catégorisation de textes, corpus de résumé
automatique, résumé automatique, traduction automatique, profilage d'utilisateurs,
recommandation d'articles de presse.  

%%~~~~~~~~~~~~~~~~~~~~~~~~~~~~~~~~~~~~~

\begin{arab}
مع بروز تكنولوجيا الإعلام و الإتصال و دمقرطة محتوى صفحات الأنترنت، أصبح من الصعب للغاية ... يدويا أو حتى باستعمال أنظمة معلوماتية.

وتساهم الصحف و المواقع الإخبارية بشكل مباشر في تزايد المحتوى، مما يجعل الإطلاع على الاخبار أمرا متعبا يتطلب الكثير من الوقت. 

لمعالجة هذه الإشكالية، نقترح في مشروعنا تطبيقا يأخذ بعين الإعتبارإههتمامات المستخدمين بهدف اقتراح مقالات من الصحافة الجزائرية و الدولية باللغتين العربية و الإنجليزية تلبي رغبة القارئ، و هذا باستخدام تقنيات الذكاء الإصطناعي و المعالجة الآلية للنصوص.

الكلمات الدالة : المعالجة الآلية للغات،  تصنيف النصوص، تلخيص النصوص، الترجمة الآلية. 

%%~~~~~~~~~~~~~~~~~~~~~~~~~~~~~~~~~~~~~

With the emergence of communication technology and the democratization of information on the internet, it has become very difficult to control manually or even by using IT (or Computer) systems .....
?? Magazines ans online newspapers contribute in a direct way (or directly) in the increase of this content (or the increase of information content), which mqkes it tedious and very difficult to stay updated with changes in our world.
Reading magazines ans newspapers become very time consuming and over whelming, considering the huge amount of content online and the speed which this content increase (and increasing)
We take in considearation (or we consider) the interests of the users in order to suggest articles from the  ....
that meets the needs and expectations of readers, for this purpose we use Artificial Intelligence techniques (or technologies) ranging from machine learning to natural language processing.

\end{arab}