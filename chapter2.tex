%!TEX program=luatex
\documentclass{report}
\usepackage[T1]{fontenc}
\usepackage[francais]{babel}
\usepackage[Lenny]{fncychap} %Sonny, Lenny, Glenn, Conny, Rejne, Bjarne, Bjornstrup
\usepackage{fontspec}
\usepackage{wrapfig}
\usepackage{graphicx}
\usepackage{soul}
\usepackage[colorlinks=true, linkcolor=blue]{hyperref}
\usepackage[a4paper, width=150mm, top=25mm, bottom=25mm]{geometry}
\usepackage{parskip}
\usepackage{enumitem}
\usepackage{titlesec}
\usepackage{listings}
\usepackage{float}
\usepackage[final]{pdfpages}
\setlist[itemize]{label=\textbullet}
\usepackage{fancyhdr}
\pagestyle{fancy}
\fancyhead{}
\fancyhead[C]{\leftmark}
\renewcommand{\headrulewidth}{0.4pt}
\renewcommand{\footrulewidth}{0.4pt}

\lstset{
language=Python,                   % choix du langage (de programmation).
keywordstyle=\color{blue},      % choix de la couleur des mots clés.
stringstyle=\color{red},        % choix de la couleur des string.
commentstyle=\color{green},     % choix de la couleur des commentaire.
basicstyle=\normalsize,     % taille de la police du code
% numbers=left,                   % placer le numéro de chaque ligne à gauche (on peut choisir à droite, ou ne pas mettre cette option pour aucun numéro de ligne).
numberstyle=\normalsize,        % taille de la police des numéros.
numbersep=0pt,                  % distance entre le code et sa numérotation.
showstringspaces=false,         % pour ne pas afficher les espaces comme des caractères .
breaklines=true,                % couper la ligne si la ligne du code est trop longue.
}

\usepackage{xcolor}
\definecolor{light-gray}{gray}{0.90}
\newcommand{\code}[1]{\colorbox{light-gray}{\texttt{#1}}}

\usepackage{listings}
\lstdefinestyle{code}{
language=python,                   
keywordstyle=\color{blue},      
stringstyle=\color{blue},        
commentstyle=\color{gray},     
basicstyle=\small\ttfamily,           
numbers=left,                   
numberstyle=\normalsize,        
numbersep=7pt,                  
showstringspaces=false,         
breaklines=true,                
frame=leftline,                 
framerule=2pt,
}

\begin{document}
% \includepdf{Page_garde.pdf}
\tableofcontents
\pagenumbering{arabic}

\chapter{Traitement automatique du langage naturel} 
\newpage
\section{Introduction}

\section{Définition}
Le TALN\footnote(Traitement automatique du langage naturel) est l'un des principaux domaines de l'intelligence artificielle, il permet de développer des systèmes qui peuvent \emph{analyser} et \emph{comprendre} le langage humain.\\
En dehors des opérations courantes de traitement de texte qui considèrent les textes comme une simple séquence de symboles, l'utilisation des techniques de TALN permet d'organiser et de structurer des quantités énormes de connaissances pour développer des applications très avancées telles que la synthèse automatique, la reconnaissance d'entités, l'analyse des sentiments, la reconnaissance de la parole...\\
Le TALN est considéré comme un problème difficile en informatique. Le langage humain est rarement précis ou simplement parlé. Comprendre le langage humain, c'est comprendre non seulement les mots, mais aussi les concepts et comment ils sont liés pour créer du sens. Bien que la langue soit l'une des choses les plus faciles à apprendre pour les humains, l'ambiguïté du langage est ce qui rend le traitement du langage naturel difficile à maîtriser pour les ordinateurs.\\
%an NLP expert at Meltwater Group, said in How Natural Language Processing Helps Uncover Social Media Sentiment. https://blog.algorithmia.com/introduction-natural-language-processing-nlp/


\section{Domaines d'applications}
Le TALN est utilisée pour manipuler le langage humain, qu'il s'agisse d'extraire du sens ou de générer du texte dans le but d'accomplir des tâches tels que le résumé automatique d'un document, la traduction entre deux langages naturels ou la détéction des spams.\\
On peut distinguer deux grande catégories où les techniques du traitement automatique de la langue ont pu révolutionnées et faire leur preuve, la recherche scientifique et l'industrie informatique.\\

    %Recherche scientifique
Dans les laboratoires de recherches en intelligence artificielle, le TALN est considéré, souvent, comme l'une des branches les plus importantes et les plus productives. De nombreuses activités cognitives qui se produisent dans l'esprit humain ont pu être simuler grâce au TALN.
On peut trouvé: 

    \subsection*{La traduction automatique}
    La traduction automatique des textes est probablement l'un des domaines les plus connu de l'IA\footnote{Intelligence artificielle}, Elle a fait l'objet de plusieurs travaux depuis très longtemps. Le processus de traduction est découpé en plusieurs phases successives. Tout d'abbord la compréhension et l'assimilation, la déverbalisation et la conservation du sens, ensuite la réexpression et la reformulation en langue cible.\\
    Le traducteur automatique le plus utilisé sur internet est \emph{Google Translate} developpé par le département de traduction automatique de Google en 2006, il supporte maintenant plus de 103 langues.\\  

    \subsection*{Le résumé automatique}
    La construction automatique de résumés est un champ de recherche originale au sein de l'informatique, même si son ampleur n'a jamais été aussi importante que la traduction automatique.\\
    Plusieurs approches ont été proposées, premièrement des systèmes qui permettant l'élaboration automatique de résumés à partir de l'extraction de phrases. Ensuite, et avec le développement des outils informatiques (logiciels et matériels), la construction de résumés s'est basé sur le fait de donner au programme informatique la capacité d'élaborer des abstractions à partir de la \emph{compréhension} de ces textes.\\

    \subsection*{La classification de texte/document}
    La classification automatique de documents/textes est un problème connu en informatique, il s'agit d'assigner un document/texte à une ou plusieurs catégories ou classes. Le problème est différent selon la nature des documents/textes en question.\\
    L'idée générale consiste en l'identification et l'extraction des éléments pertinents à partir d'un texte/document contenant des informations dont la nature est spécifiée à l'avance. Elle vise donc à transformer un texte de son format initial (une suite de chaines de caractères) à une représentation structurée et donc un format qui soit compréhensible par l'ordinateur.\\


    %Industrie
Dans l'industrie, et avec le coût du calcul qui ne cesse de baisser, l'évolution exponentielles des algorithmes et surtout la disponibilité ... des données sur les différents supports numériques, les entreprises ont commencé à s'intérésser à l'analyse et l'exploitation de ces quantités massives de connaissances.
Grâce au TALN on a pu trouver des réponses aux différentes questions fréquentes. 

    \subsection*{Service Client: "Comment puis-je garder mon client heureux ?"}
    Fortement utilisées dans le service client, les techniques de TALN permettent de développer des systèmes capables de simuler les interactions entre les clients et les entreprises,
    Ces derniers pointent vers les raisons de l'insatisfaction des consommateurs..........\\
    % Afin de garder le pouls de l'intention des consommateurs, d
    De nombreuses entreprises analysent maintenant les enregistrements d'appels clients, les discussions sur les réseaux sociaux et ..track. les comportements de leurs clie...., et déploient des robots de discussion et des assistants en ligne automatisés pour fournir une réponse immédiate aux besoins simples et réduire la charge de travails pour leurs employés.\\ 

    Les tâches TAL pertinentes incluent:\\
    \begin{itemize}
        \item \textbf{La reconnaissance vocale:} convertit le langage parlé en texte. Les progrès de l'apprentissage profond (Deep Learning) au cours des 10 dernières années et les quantitées massives de données disponibles sur internet ont permis de déployer cette technologie dans des systèmes commerciaux tels que Siri d'Apple, Alexa d'Amazon et Google Assistant/Home dernierement.\\
        \item \textbf{Système de Question/Réponses:} répondre aux questions posées par les humains dans une langue naturelle. La technologie est utilisée aujourd'hui par de nombreuses entreprises pour les chatbots, à la fois pour les projets internes (RH, opérations) et externes (service client). Ces systèmes sont implémentés, pratiquement en natif, sur tout les systèmes d'exploitations mobile (Android, IOS)  
    \end{itemize}

    \subsection*{E-réputation: "Que disent les gens à propos de moi ?" }
    Les entreprises ont commencé, et cela depuis les années 80, à utiliser des logiciels pour trouver des modèles dans leurs propres données et prendre de meilleures décisions.\\ 
    L'optimisation des chaînes d'approvisionnement, des inventaires et des entrepôts, des processus de vente et de nombreuses autres applications ont donné naissance à ce que nous appelons aujourd'hui le Business Intelligence (BI). Mais pour une entreprise le plus important et le plus précieux est ce qui est dit dehors. C'est ce qui a poussé ces derniers à adopter des outils leur permettant d'exploiter les données externes/publiques collectées sur les réseaux sociaux.\\
    Certaines de ces données sont structurées et prêtes à être analysées, la plus grande partie générés par l'homme tels que les articles de blog, commentaires sur les forums ou les offres d'emploi reste non structurées. Ces sources contiennent des informations précieuses sur l'évolution des concurrents, des clients et du marché dans son ensemble.\\
    Selon l'enquête BrightLocal..., 92\% des clients lisent les avis en ligne et 86\% n'achèteront pas un produit avec moins de 3 étoiles sur 5, ce qui confirme que la plupart des clients vérifient les avis en ligne avant d'acheter un produit quelque soit son prix. 
    Et comme les consommateurs formulent leurs plaintes de plus en plus sur Facebook et Twitter, la surveillance et la gestion de la e-réputation sont devenues une priorité pour les entreprises.\\

    Tâches TAL pertinentes pour cette application comprennent:
    \begin{itemize}
        \item \textbf{L'analyse de sentiment:} déterminer l'attitude, l'état émotionnel, le jugement ou l'intention de l'internaute (positive, neutre ou négative) ou aussi reconnaitre l'humeur (heureux, triste, calme, en colère ...).\\
    \end{itemize}

    \subsection*{Publicité: "Qui est intéressé par mon produit ?"}
    Les emails, les médias sociaux, le commerce électronique et les comportements sur les navigateurs contiennent beaucoup d'informations sur ce qui nous intéresse vraiment. L'énorme potentiel de ce type de données non structurées est confirmé par le fait les plus grandes entreprises génèrent aujourd'hui le plus de de leurs recettes de vente d'annonces (Google et Facebook).\\ 
    Tâches TAL pertinentes pour cette application comprennent:

    \begin{itemize}
        \item \textbf{Correspondance par mot-clé (matching):} vérifie si des mots d'intérêt sont inclus dans un texte. 
        \item \textbf{Désambiguïsation:} identification du sens d'un mot utilisé dans une phrase.
    \end{itemize}
    
\section{Techniques du TALN}
Afin de réaliser une des applications sus-citées, un phase de pré-traitement des données (texte, phrase, mot...) est nécessaire. C'est une étape qui nécessite de l'expertise et la maîtrise de plusieurs techniques, et c'est l'étape  

\subsection{Expressions régulières}
Une expression régulière est une chaîne de caractères, qui décrit, selon une syntaxe précise, un ensemble de chaînes de caractères possibles. Elles sont utilisées pour programmer des logiciels avec des fonctionnalités de lecture, de contrôle, de modification, et d'analyse de textes ainsi que dans la manipulation des langues formelles que sont les langages informatiques.\\
On peut les retrouver dans plusieurs utilitaires tel que \textbf{GNU grep} qui utilisent ces expressions pour parcourir de façon automatique un document à la recherche de morceaux de texte compatibles avec le motif de recherche, et éventuellement effectuer un ajout, une substitution ou une suppression.

L'Expression régulière suivante permet la reconnaissance des adresses mails:\\
$$[\\w+.-]+@[\\w.-]+\\.[a-zA-Z]\{2,\}$$

    
\subsection{Segmentation (Tokenization)}
C'est l'opération la plus basique dans un process de TALN, elle consite en l'identification des tokens (mots), ou de phrases entières dans un texte que nous voulons traiter. La difficulté est dans le fait que l'utilisation de la ponctuation et les séparateurs pour la segmentation, et dans plusierus langues dont l'anglais et l'arabe, est souvent ambigu.     
De nombreux algorithmes de segmentation appelés \emph{Tokenizer} sont disponibles sur internet en libre accés:
\begin{itemize}
    \item \textbf{RegexpTokenizer} 
    \item \textbf{Word_tokenize}
    \item \textbf{TweetTokenizer} 
    \item \textbf{PTBTokenizer}
    \item
\end{itemize}




\subsection{Lemmatisation et racinisation}
La racine d’un mot correspond à la partie du mot restante une fois que l’on a supprimé son préfixe et son suffixe (et infixes dans certaines langue comme l'Arabe), à savoir son radical. Elle est aussi parfois connu sous le nom de stemme d’un mot.\\ 
Contrairement au lemme qui correspond à un mot réel de la langue, la racine ou stemme ne correspond généralement pas à un mot réel.

\textbf{Lemmatisation:} "chercher" => "cherch"\\
\textbf{Racinisation (stemming):} "frontalier" => "front"  

Plusieurs outils destinés à la lemmatisation et la racinisation sont implémenté dans des librairies majoritairement \emph{open source} et dans différents languages de programmation. Parmis ces dernieres on peut cités:
\begin{itemize}
    \item \textbf{Porter Stemmer} développé en 1979 par \emph{Martin Porter}  à Cambridge (Angleterre). 
    L'algorithme permet d'éliminer les terminaisons morphologiques des mots en anglais.
    \item \textbf{Snowball stemmer} mis en place par un groupe de linguiste, il prend en charge officielement 14 langue dont l'Anglais et le Français. 
    \item ....other
\end{itemize} 


\subsection{L'étiquetage morpho-syntaxique (POS Tagging)}
Le POS Tagging est le processus qui consiste à associer à chaque mot d'un texte les informations grammaticales correspondantes comme la partie du discours, le genre, le nombre, etc. avec l'utilisation des programmes informatiques.\\
L'étiquetage morpho-syntaxique est une opération très complexe, le fait d'avoir des mots et leur étiquettes est souvent pas suffisant vu les ambiguités qu'on peut rencontrer (pour un même mot, différentes étiquettes possible)
Exemple:\\
Le paysan ferme la ferme
DET NN V DET NN
DET NN ADJ PRN V

Pour la langue anglaise on peut distinguer entre 50 et 150 étiquettes morpho-syntaxique selon le besoins et la précision voulues.
Il existe également, comme pour les Stemmer, un grand ensemble d'algorithmes de POS Tagging (pré)entrainé:
\begin{itemize}
    \item Treetager
    \item ........
    \item Stanford Log-linear Part-Of-Speech Tagger
\end{itemize}

\section{Aspect du langage}
    \subsection{Corpus}

    \subsection{Base lexicales}
        \subsubsection{WordNet}


\section{Résumé automatique}
    \subsection{Définition}
    \subsection{Domaines d'applicaions}

    \subsection{Résumé extractif}

    \subsection{Résumé abstractif}


\section{Catégorisation}
    \subsection{Catégorisation d'articles de presse}


\section{Système de traduction automatique}


\section{Conclusion}


\Large
\bibliographystyle{unsrt}
\bibliography{biblio}
\end{document}
