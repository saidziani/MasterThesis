%!TEX program=luatex

Aujourd'hui, les internautes se retrouvent quotidiennement face à un flux d'informations massif et une diversité de contenu très importante; les médias et sites d'actualité y contribuent pleinement. Dans un environnement où l'utilisateur ne dispose toujours pas de moyen satisfaisant pour gérer ce flux, un nouveau besoin se crée, celui d'une expérience de lecture personnalisée, correspondant le plus pertinemment et efficacement aux besoins et centres d'intérêt de l'utilisateur.
 
La première motivation de ce travail a été de fournir un outil qui permet au lecteur de faire sa revue de presse quotidienne de la façon la plus efficace et la plus enrichissante possible. Nous réalisons à présent qu'entreprendre ce projet s'étend bien au-delà du cadre d'un projet de fin d'études et s'inscrit dans une optique d'innovation, dont l'objectif est d'apporter une réelle contribution et une valeur ajoutée à toutes les solutions existantes.
 
Dans le présent mémoire, nous avons examiné en premier lieu la littérature des systèmes de recommandation tout en explicitant les différents concepts liés au profilage utilisateur, Et ce afin d'enlever l'ambiguïté quant à ses différents usages pour bien cerner les possibilités d'exploitation de ses techniques.
 
Nous avons ensuite effectué un long travail de recherche sur l'état de l'art du traitement automatique du langage naturel. Nous avons ainsi identifié les travaux les plus fiables, les différentes techniques utilisées et les outils les plus performants.
 
Une longue période du projet a été dédiée à la recherche, la récolte et le pré-traitement des données. Une fois les datasets de chaque modules de l'application créés, une étude exploratoire a été effectuée sur ces derniers afin de détecter les composants homogènes des données disponibles et de mettre en évidence les indicateurs discriminants des instances des corpus.

Néanmoins, un des principaux freins au projet a été le manque flagrant de références, de corpus et d'outils pour le traitement automatique de la langue arabe. Un travail conséquent a été effectué pour le pré-traitement et la préparation des datasets, liés à l'aberration et la particularité de cette langue.
 
Nous avons également présenté la conception de la solution proposée, détaillé les différentes démarches suivies et défini les approches adoptées, tout en justifiant les choix entrepris, leurs avantages et leurs inconvénients.
 
La dernière phase aborde l'implémentation et l'intégration des différents modules, dans laquelle nous avons présenté une étude comparative des modèles d'apprentissage automatique développés pour chacune des quatre problématiques principales : la catégorisation, le résumé automatique, la traduction automatique et la recommandation d'articles de presse. Nous avons par la suite présenté les résultats obtenus et évalué ces derniers en utilisant les mesures et les métriques adéquates.
 
Les résultats de la classification d'articles de presse ont été très satisfaisants comparés aux travaux similaires. Les modèles inférés pour la prédiction des catégories ont donné un taux de précision de 99\%. Nous avons également eu de très bons résultats quant au résumé automatique dans les deux langues.
 
La dernière étape de notre projet a été la réalisation de l'application \textquotedbl Feedny\textquotedbl.\\
\textquotedbl Feedny\textquotedbl est une application mobile qui intègre tous les modèles développés. Elle permet au lecteur d'exploiter toutes les fonctionnalités présentées précédemment. L'expérience utilisateur et le design des interfaces ont été conçus de telle sorte à compléter les fonctionnalités par l'intuitivité et le plaisir de l'utilisation.
 
La dernière partie du mémoire présente les fonctionnalités et interfaces de l'application ainsi que les différentes phases du développement en justifiant le choix des outils et les langages de programmation.
 
Nous pouvons aujourd'hui dire que notre objectif a été atteint, puisque l'application que nous avons réalisée satisfait largement les principales ambitions définies.
 
Ce projet nous a été bénéfique sur plusieurs plans. Il nous a permis de mettre en pratique toutes les connaissances acquises lors de notre cursus de Master en Systèmes Informatiques Intelligents, de développer une méthodologie de recherche et de distinction des travaux, d'élargir les notions théoriques dans le TALN, l'apprentissage automatique et la recherche d'informations, et de nous initier aux systèmes de recommandations d'articles de presse. Le mémoire nous a permis d'améliorer notre capacité rédactionnelle et de devenir plus rigoureux quant aux questions de plagiat, des sources d'informations et de leur référencement. Le travail en binôme, supervisé par notre promotteur a renforcé notre capacité à travailler en équipe.
 
Toutefois, il est très important de signaler que la solution proposée reste largement perfectible. Plusieurs points demandent encore à être revus et retravaillés. L'une des perspectives qui pourraient s'avérer intéressantes serait de construire un large corpus pour le résumé automatique abstractif, afin de donner plus d'exactitude aux résumés générés, l'approche abstractive dans le résumé automatique étant elle-même très complexe. Il serait très utile aussi d'ajouter d'autres catégories d'articles de presse, ce qui donnera plus de diversité au contenu proposé. Les modèles de classification d'articles pourraient être améliorés afin de s'adapter aux vocabulaires récents et de prendre en charge les nouveaux mots du langage (Brexit, Facebooker, etc.).

% Notre système de recommandation doit être évalué dans le future, dés que le nombre d'utilisateurs et interactions le permettenet
Quant à l'application \textquotedbl Feedny\textquotedbl, elle pourrait être enrichie également par d'autres fonctionnalités telles que l'ajout de nouvelles sources d'articles de presse par l'utilisateur. Mais aussi par le fait de proposer plusieurs sources pour un même sujet d'article afin d'identifier les sources les plus fiables et objectives.
Et bien sûr, l'application ne gagnerait que plus d'utilisateurs et ne deviendrait que plus accessible si une plateforme web de la solution est développée.