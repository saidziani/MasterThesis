%!TEX program=luatex

\vspace{1.5cm}

\setlength{\parindent}{0.5cm}
Avant tout, je remercie ALLAH le tout-puissant de m'avoir donné le courage, la volonté et la patience de mener à terme ce présent travail dans les meilleures conditions.

Je tiens à exprimer toute ma reconnaissance à mon encadreur principal, Monsieur Ahmed GUESSOUM. Je le remercie de m’avoir encadré, orienté, aidé et conseillé, et pour la liberté de travail qu'il m'a laissée tout au long de ce semestre. 
Je le remercie aussi pour sa disponibilité, sa patience et surtout ses judicieux conseils, qui ont contribué à alimenter ma réflexion.

J’adresse mes remerciements à mon Co-Encadreur, Monsieur Riadh BELKEBIR, pour ses paroles, conseils et critiques.

J'adresse également, mes remerciements à Madame F. KHELLAF qui nous fait l'honneur de présider le jury de notre soutenance. Mes remerciements s’adressent également à Madame N. ABDAT pour avoir accepté de faire partie de ce jury.

J’adresse mes sincères remerciements à tous les professeurs de l'USTHB , intervenants, et toutes les personnes qui par leurs paroles, leurs écrits, leurs conseils et leurs critiques ont guidé mes réflexions et ont accepté de me rencontrer et répondre à mes questions durant mon cursus et mes recherches et qui doivent voir dans ce travail la fierté d'un savoir bien acquis.

Je remercie mes très chers parents, qui ont toujours été là pour moi. Ma mère, qui a œuvré pour ma réussite, de par son amour, son soutien, tous les sacrifices consentis et ses précieux conseils, pour toute son assistance et sa présence dans ma vie : reçois à travers ce travail aussi modeste soit-il, l'expression de mes sentiments et de mon éternelle gratitude. Mon père, qui peut être fier et trouver ici le résultat de longues années de sacrifices et de privations pour m'aider à avancer dans la vie. Puisse Dieu faire en sorte que ce travail porte son fruit ; merci pour les valeurs nobles, l'éducation et le soutien permanent venu de toi.

« Vous avez tout sacrifié pour vos enfants n’épargnant ni santé ni efforts. Vous m’avez donné un magnifique modèle de labeur et de persévérance. Je suis redevable d’une éducation dont je suis fier ».
Je remercie mon frère et ma sœur pour leur encouragement. Ma pensée va également à mon grand père, et toute ma famille qui m'ont encouragé pendant ce travail.

Je tiens à  remercier en particulier Nadjib et Billel pour leur appui lors de mes stages en externe durant tout mon cursus.
J'adresse tous mes voeux de réussite à Saïd ZIANI (mon binôme), ainsi que tous les autres étudiants en Master 2 SII spécialement et les autres Masters en général.

Enfin, je remercie tous mes Amis que j’aime tant, Karim, Nidal, Salim, Haithem, Houssem, Driss, Nabil, Said, Lahcen, Anis, Walid, Hichem, Ilyes, Younes, Ayoub, Rafik et tous mes autres ami(e)s pour leur sincère amitié et confiance, et à qui je dois ma reconnaissance et mon attachement.

À tous ces intervenants, je présente mes remerciements, mon respect et ma gratitude.

\vspace{0.5cm}
\begin{center}
\Large
\hspace{12.5cm}
\textbf{Fawzi}
\end{center}