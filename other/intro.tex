%!TEX program=luatex

\newpage

Voici vingt ans que les premiers journaux occidentaux (américains et européens) ont commencé la diffusion de l'information sur le web. Les premiers sites web des quotidiens d'information sont apparus en 1995. Rapidement, toutes les organisations de presse se sont tournées vers le média en ligne, Internet.

Dès les années 90, de nombreux sites diffusent des informations d'actualité alors qu'ils n'ont rien à voir avec le journalisme. Dans la même décennie, plusieurs formes de sites d'informations sont nées, entre blogs, webzines (magazines en ligne) ou sites indépendants.

De nos jours, et avec l'émergence des Technologies de l'Information et de la Communication et la démocratisation de la production de contenu sur internet, le nombre de sites d'informations et les quantités énormes de contenu créé ne cessent d'augmenter exponentiellement chaque jour.

En Algérie, comme partout dans le monde, les acteurs médiatiques, notamment des journaux et des chaînes télévisées, ont vu leurs canaux de communication muer et leur digitalisation se développer progressivement au point où certains utilisent le web comme unique médium. On peut citer TSA (Tout Sur l'Algérie), Algérie Focus, Casbah Tribune, etc.

Avec toutes ces quantités informationnelles et ce nombre croissant de médias en ligne, le lecteur algérien, ou autre, se trouve incapable de traiter ce flux d'informations manuellement ni même avec des outils technologiques (classiques). 

// et mêmeles outils technologiques (classiques) ne remédie pas à cette provlématique //

C'est la raison principale pour laquelle il a été jugé utile de concevoir et de réaliser un outil qui permet à l'utilisateur d'exploiter, de la façon la plus efficace possible, toutes ces ressources en se basant sur ses préférences.

L'outil que nous proposons utilisera des techniques d'intelligence artificielle, de traitement automatique du langage, d'apprentissage automatique, de recherche d'information et de recommandation personnalisée. En se basant sur ces derniers, notre solution sera capable d'aider le lecteur à faire sa revue de presse quotidienne de la manière la plus enrichissante possible, mais aussi la plus optimale. Et en utilisant des techniques de profilage intelligent, les recommandations d'articles de presse seront adaptées aux centres d'intérêts et préférences de chaque utilisateur, et seront plus précises avec le temps grâce aux évaluations des suggestions précédentes.

Et dans l'optique de rendre cet outil plus sophistiqué, nous proposons à l'utilisateur algérien //à un utilisateur arabophone// un accès efficace à un contenu anglophone avec possibilité de traduction automatique. Nous mettrons en place également une fonctionnalité de génération de résumé automatique pour chaque article, ce qui sera d'une grande aide à toute personne voulant gagner du temps, notamment les professionnels du domaine journalistique, politique ou des affaires, devont faire// la veille de toutes les dernières actualités.

Pour ce faire, nous étudierons les concepts théoriques de base de chaque technique tout en examinant la littérature et les travaux de l'état de l'art. Nous allons, ensuite, travailler sur la conception des différents modules du système. Et enfin, nous passerons à la réalisation qui sera suivie par //des//tests et évaluations.

Le premier chapitre de ce mémoire consistera en la présentation des systèmes de recommandations. Le deuxième portera sur le  traitement Automatique du Langage Naturel. Le troisième, quant à lui, fera l'objet de l'étude conceptuelle. Enfin, dans le quatrième et dernier chapitre, nous présenterons le produit final ainsi que sa réalisation. Puis nous parlerons des perspectives possibles et de l'avenir de l'application avant de dresser une conclusion générale du projet.

% Le présent travail est ainsi divisé en quatre chapitres. 
% \begin{itemize}[leftmargin=*,label={-}]
%     \item Le chapitre 1 s'intitule Les systèmes de recommandations.
%     \item Le chapitre 2 porte sur le traitement Automatique du Langage Naturel.
%     \item Le chapitre 3, quant à lui, décrit notre conception.
%     \item Et enfin, le chapitre 4 présente la réalisation.
% \end{itemize}